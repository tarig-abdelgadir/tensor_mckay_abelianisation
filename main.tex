\documentclass{amsart}
\usepackage[utf8]{inputenc}

\usepackage{mymacros, stmaryrd, tikz, pstricks, pst-node, pst-text, pst-tree, subfigure, epsfig, psfrag}
\usepackage{float}

\newtheorem{thm}{Theorem}[section]
\newtheorem{cor}[thm]{Corollary}
\newtheorem{lem}[thm]{Lemma}
\newtheorem{prop}[thm]{Proposition}

\theoremstyle{definition}
\newtheorem{defn}[thm]{Definition}
\newtheorem{rem}[thm]{Remark}
\newtheorem{notn}[thm]{Notation}
\newtheorem{conj}[thm]{Conjecture}
\newtheorem{eg}[thm]{Example}

\title{Tensor McKay: SL(2)}
\author{Tarig Abdelgadir}
\author{Joseph Karmazyn}
\date{\today}

\begin{document}

\maketitle

\section{Introduction}
Balmer's result maybe interpreted as saying that derived equivalences between varieties are variations in tensor structure.

This is repackaging of known results but we feel that viewing them under this light is potentially useful. 

\section{Background and Notation}

We fix a finite subgroups $G$ of $\SL(2,\bC)$.
Such subgroups of $\SL(2,\bC)$ are labelled by simply laced Dynkin diagrams.
We will label the root system and affine root system corresponding to $G$ by $\Delta$ and $\tilde{\Delta}$ respectively.
Furthermore, we pick a set of simple roots for $\Delta$ and call them $\{\alpha_i\}$.
The addition of the standard imaginary root $\alpha_0$ gives us an induced root system on $\tilde{\Delta}$.
Entries of the corresponding generalised Cartan matrix will denoted
$$c_{ij} := 2 \frac{\langle \alpha_i, \alpha_j\rangle}{\langle \alpha_j, \alpha_j\rangle}.$$

We take $(Q,I)$ to be the McKay quiver with relations associated to $G$.
One may identify the vertices of $Q$ with the simple roots $\alpha_i$ of $\tilde{\Delta}$.
This, in particular, gives a distinguished vertex: namely the one corresponding to $\alpha_0$, we will call it $0$.
The minimal positive integer vector in the kernel of the Cartan matrix gives us a dimension vector $\bfv=(v_i)$.
We pick $\theta$ a stability parameter in the chamber corresponding to 0-generated stability conditions, for example $(-\sum_{i \neq 0} v_i,1, \ldots,1)$.
The moduli space $\theta$-stable quiver representations of $Q$ of dimension vector $\bfv$ satisfying the relations given by $I$ will be denoted $\cM^\theta$.
Note that this is naturally isomorphic to $G$-Hilb$(\bC^2)$.
The space $\cM^\theta$ comes with tautological vector bundles $T_i$, one for each vertex.
Our construction of $\cM^\theta$ ensures that $T_0$ is the trivial line bundle.
We define $L_i$ to be the top wedge of the vector bundle $T_i$, i.e.\ $L_i := \det(T_i)$.

The vector bundles $T_i$ generate the derived category of $\cM^\theta$.
The following is a shirt exact sequence $$0 \rightarrow T_0^{\oplus (v_i-1)} \rightarrow T_i \rightarrow L_i \rightarrow 0.$$
Therefore, the bundles $L_i$ generate the derived category too.
We have a derived equivalence between $Y$ and $\bk Q/I$ taking $T_i$ to the projective at the corresponding vertex $P_i$.

\section{Tensor structure on $\cM^\theta$}

\begin{prop}
The following complex is exact: \[
0 \rightarrow L_0 \rightarrow L_i \oplus L_j \rightarrow L_i \otimes L_j \rightarrow 0.
\]
\end{prop}

\begin{proof}
This will enventually follow from the main result of this section.
\end{proof}

This is however boring. It doesn't tell us anything about VGIT.

For the purposes of variation of stability parameter the building blocks of the complex above.

We have a Hasse diagram of simple reflections in our Weyl group with $S_0$ as a quotient. 
We also have a Hasse diagram of positive roots of the form $\alpha_0+\beta$ where $\beta \in \Lambda^+$. 
These two Hasse diagram are combinatorially equivalent.
We match them up to give us the following exact triangles
$$\bigotimes_{\alpha_i<0} L_i^{\alpha_i} \longrightarrow \bigotimes_{\alpha_j>0} L_j^{\alpha_j} \longrightarrow \alpha(S_0) \longrightarrow \bigotimes_{\alpha_i<0} L_i^{\alpha_i}\,[1].$$

Exactness of these triangles comes from the fact that this what happens to the trivial line bundle as I apply spherical twists going around

The other chambers in the quiver stability conditions are Weyl chambers.
A Weyl group element acts on $\alpha$ and by spherical twists on $\alpha(S_*)$ and we get another sequence.

The triangle should be seen as expressing the left most term in terms of the other terms.
For a purely geometric perspective we have that these morphisms are composites.

Why is this enough? It's the tensor product of non-mutually vanishing sections.

Why do they all show up? Because the top of their respective pyramids will have then in the right place.

The simple reflections give us complexes of the form:
$$ 0 \rightarrow \otimes_{j} \rho_j \otimes \rho_i^\vee \rightarrow \oplus_{j} \rho_j \rightarrow \rho_i \rightarrow 0$$ 
or an exact triangle
$$\rho_i \rightarrow \otimes_{j} \rho_j \otimes \rho_i^\vee \rightarrow S_i \rightarrow \rho_i[1].$$
We will rigidify so that the trivial bundle is the unit. 
The spherical twists to the side give the functor tensoring by the dual of the line bundle.

The spherical twists applied to the simple at the star vertex form a Hasse diagram.

\begin{lem}
The following is an exact triangle in the derived category
$$L_i \otimes L_j \rightarrow L_i \boxtimes L_j \rightarrow \sigma(S_*) \dashrightarrow L_i \otimes L_j\,[1].$$
\end{lem}

How does one build from the simple roots?

The Hasse diagram also builds the extensions in a certain way, these will form the cones that we will use to describe the tensor structure.
Note that these extensions are dependent on the stability parameter $\theta$.
The determinants of the tautological bundles generate the category in question so it suffices to describe their mutual tensor products.
The structure morphisms given by the walls of a given chamber completely describe the tensor product we want.

Something about conjugating the Hasse diagram with the element of the reflection group in question.

\section{Variation of tensor structure}

\section{Moduli of tensor representations}

We can look at the GIT problem given by using these structure maps as fields and act on them by the Gauge group.
The point is that this will end up giving us the canonical bundle over the corresponding weighted projective lines in one chamber while we have the crepant resolution in another.

The GIT problem on the one hand recovers the canonical bundle over the corresponding weighted projective line a partial resolution of the singualrity and the minimal resolution on the other.

\section{Moduli of tensor stable representations}

\end{document}
